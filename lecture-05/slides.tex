\documentclass{beamer}


\mode<presentation>
{
  \usetheme{CambridgeUS}
	\usecolortheme{beaver}
  % or ...

  \setbeamercovered{transparent}
  % or whatever (possibly just delete it)
}


\usepackage{xeCJK}
\usepackage{ulem}
\usepackage[english]{babel}
\usepackage[utf8]{inputenc}
\usepackage{times}
\usepackage[T1]{fontenc}
\usepackage{hyperref}
\usepackage{pifont}
\usepackage{biblatex}
\usepackage{bibentry}
\usepackage{verbatim}
\bibliography{cite}
\newcommand{\cmark}{\ding{51}}%
\newcommand{\xmark}{\ding{55}}%
\setCJKmainfont{WenQuanYi Micro Hei}
\renewcommand{\raggedright}{\leftskip=0pt \rightskip=0pt plus 0cm}
\raggedright

\let\oldfootnotesize\footnotesize
\renewcommand*{\footnotesize}{\oldfootnotesize\tiny}

\title[Intelligent Software Engineering] 
{Case Study of Software Engineering Research}
\subtitle{What We Talk About, When We Talk About Reproducible Builds}

\author[Zhilei Ren] 
{Zhilei Ren}

\institute[Dalian University of Technology] % (optional, but mostly needed)
{
\\\includegraphics[width=0.1\textwidth]{../utils/logo.png}\\
Dalian University of Technology
}


\subject{Software Engineering}



\pgfdeclareimage[width=0.08\textwidth]{university-logo}{../utils/logo.png}
\logo{\pgfuseimage{university-logo}}



% Delete this, if you do not want the table of contents to pop up at
% the beginning of each subsection:
\AtBeginSubsection[]
{
  \begin{frame}<beamer>{Outline}
    \tableofcontents[currentsection,currentsubsection]
  \end{frame}
}


% If you wish to uncover everything in a step-wise fashion, uncomment
% the following command: 

%\beamerdefaultoverlayspecification{<+->}

\setbeamertemplate{section in toc}[circle]
\setbeamertemplate{items}[circle]
\setbeamertemplate{caption}[numbered]
\setbeamertemplate{bibliography item}{\insertbiblabel}
\setbeamertemplate{bibliography entry title}{}
\setbeamertemplate{bibliography entry journal}{}

\begin{document}

\begin{frame}
  \titlepage
\end{frame}

%\begin{frame}{Outline}
%  \tableofcontents[currentsection,currentsubsection, 
%    hideothersubsections, 
%    sectionstyle=show,
%]
%\end{frame}

\AtBeginSection[]
{
 \begin{frame}<beamer>
 \frametitle{Outline}
 \tableofcontents[currentsection]
 \end{frame}
}

\begin{frame}{Checksum is important for software security}
\begin{itemize}
\item Checksum provides a linkage between user downloaded software and the one provided by the repositories
\item Build environments can be compromised!
\end{itemize}
\end{frame}

\begin{frame}{XcodeGhost Case}
\begin{itemize}
\item A counterfeit version of Xcode infected 4,000 iOS apps in 2015
\item More than 200 enterprises influenced
\item Injecting malicious code during compiling time
\item \url{https://www.fireeye.com/blog/executive-perspective/2015/09/protecting_our_custo.html}
\end{itemize}
\end{frame}

\begin{frame}{For open-source software, will it be easier?}
\begin{itemize}
\item Solution: verify compiled binaries under diverse environments
\item Attacks may elude detection due to unexpected factors
\item There is still gap between source and binary
\end{itemize}
\end{frame}

\begin{frame}{Yet another case of unreproducible build}
\begin{itemize}
\item libical-dev 1.0-1.1 in Debian
\item Severity classified as "Release Critical"
\item Packages depending on libical-dev may break when rebuilt
\item Unreproducible due to non-deterministic hash table traversal
\end{itemize}
\end{frame}

\begin{frame}{Now, many distros validate package reproducibility!}
\begin{itemize}
\item Many open-source distributions have initiated validation processes
\item Debian, Guix, ArchLinux, and Bitcoin
\item As of May 2018, over 90\% of Debian's packages can be reproducibly built
\end{itemize}
\end{frame}

\begin{frame}{The unreproducible build problem}
\begin{block}{Definition}
A build is reproducible if given the same source code, build environment and build instructions, any party can recreate bit-by-bit identical copies of all specified artifacts.
\end{block}
\begin{itemize}
\item Manual work! Could it be automated?
\end{itemize}
\end{frame}

\begin{frame}{Highlights: RepLoc}
\begin{itemize}
\item An automated localization approach
\item 79.28\% top-10 accuracy ratio over 671 packages
\item We fixed 6 packages from Debian and Guix
\item 4 are accepted by the open-source community
\item Dataset publicly available: \url{https://reploc.bitbucket.io}
\end{itemize}
\end{frame}

\begin{frame}{Challenges}
\begin{block}{Challenge 1: Insufficient information}
\begin{itemize}
\item Information may not be sufficient for linking diff log with problematic files
\item No enough bug reports/test cases
\item Diff logs are not always well informative
\end{itemize}
\end{block}

\begin{block}{Challenge 2: Diverse causes}
\begin{itemize}
\item Debian rules (29.82\%) 
\item Auxiliary files (17.21\%)
\item Scripts (14.60\%)
\item Makefiles (11.68\%)
\item C/C++ files (5.94\%)
\end{itemize}
\end{block}
\end{frame}

\begin{frame}{The RepLoc framework}
\begin{center}
\includegraphics[width=0.8\textwidth]{example-image}
\end{center}
\begin{itemize}
\item Query Augmentation
\item File Ranking
\item Heuristic Filtering
\end{itemize}
\end{frame}

\begin{frame}{Query augmentation}
\begin{center}
\includegraphics[width=0.8\textwidth]{example-image}
\end{center}
\begin{itemize}
\item Basic query → Enhanced query
\item Build Command Retrieval
\item Source File Retrieval
\end{itemize}
\end{frame}

\begin{frame}{Heuristic filtering}
\begin{itemize}
\item Manually learn notes from Debian documentation
\item Capture issues as 14 Perl Compatible Regular Expression (PCRE) rules
\end{itemize}
\end{frame}

\begin{frame}{File ranking}
\begin{center}
\includegraphics[width=0.8\textwidth]{example-image}
\end{center}
\begin{itemize}
\item Combining Query Augmentation and Heuristic Filtering
\item More accurate localization of problematic files
\end{itemize}
\end{frame}

\begin{frame}{Empirical results: Experimental environment}
\begin{itemize}
\item Intel Core i7 4.20 GHz CPU
\item 16 GB RAM
\item GNU/Linux kernel 4.9.0
\item Perl 5.24, Java 1.8
\item 671 packages from Debian's BTS
\item 4 categories: Timestamps, Locale, Randomness, File-ordering
\end{itemize}
\end{frame}

\begin{frame}{RQ1: Is RepLoc sensitive to weighting parameter α?}
\begin{center}
\includegraphics[width=0.8\textwidth]{example-image}
\end{center}
\begin{itemize}
\item RepLoc is not very sensitive to α
\item Assigned value: 0.3
\end{itemize}
\end{frame}

\begin{frame}{RQ2: How effective is RepLoc?}
\begin{center}
\includegraphics[width=0.8\textwidth]{example-image}
\end{center}
\begin{itemize}
\item Achieves 79.28\% top-10 accuracy ratio
\end{itemize}
\end{frame}

\begin{frame}{RQ3: How efficient is RepLoc?}
\begin{itemize}
\item Median execution time: 5.14 seconds
\end{itemize}
\end{frame}

\begin{frame}{RQ4: Is RepLoc helpful in localizing unfixed packages?}
\begin{itemize}
\item Fixed 6 packages from Debian and Guix:
\begin{itemize}
\item regina-rexx
\item manpages-tr
\item fonts-uralic
\item skalibs
\item djvulibre
\item libjpeg-turbo
\end{itemize}
\item 4 patches accepted (1 for Debian, 3 for Guix)
\end{itemize}
\end{frame}

\begin{frame}{A case of fixed issue for djvulibre for Guix}
\begin{itemize}
\item \url{https://debbugs.gnu.org/cgi/bugreport.cgi?bug=28015}
\item Patch modifies djvu.scm to add reproducible phase
\item Guix maintainer's response: pushed with cosmetic changes
\item Issue with .svgz files not handled by new phase
\end{itemize}
\end{frame}

\begin{frame}{Conclusions and future work}
\begin{block}{Contributions}
\begin{itemize}
\item First work to address localization task for unreproducible builds
\item Effective framework integrating heuristic filtering and query augmentation
\item Fixed six previously unfixed packages
\item Dataset available at \url{https://reploc.bitbucket.io}
\end{itemize}
\end{block}

\begin{block}{Future work}
\begin{itemize}
\item More accurate approaches using static analysis or dynamic file tracking
\item Generalize knowledge to other repositories
\item Automated fixing of unreproducible issues
\end{itemize}
\end{block}
\end{frame}

\begin{frame}{Acknowledgements}
\begin{itemize}
\item Debian reproducible builds team: Holger Levsen, Chris Lamb, Ximin Luo, ...
\item Guix distribution: Ludovic Courtès, ...
\item OSCAR Lab: \url{http://oscar-lab.org}
\end{itemize}
\end{frame}

\begin{frame}{Ideas currently working on}
\begin{itemize}
\item Why RepLoc works:
\begin{itemize}
\item Heuristic rules for frequent patterns (static)
\item Build log for inconsistent-files-related commands (dynamic)
\end{itemize}
\item Why RepLoc is not good enough:
\begin{itemize}
\item Heuristic rules treat all files uniformly
\item Not all build commands captured by build logs
\end{itemize}
\end{itemize}
\end{frame}

\begin{frame}{System call interposition approach}
\begin{itemize}
\item Improve localization by analyzing system calls during build process
\item Use strace to monitor interactions between processes and kernel
\item Track dependency types: read/write operations, rename operations
\end{itemize}
\end{frame}

\begin{frame}{Process tree and dependency graph}
\begin{center}
\includegraphics[width=0.8\textwidth]{example-image}
\end{center}
\end{frame}

\begin{frame}[fragile]{System call trace example}
\begin{scriptsize}
\begin{verbatim}
2608 2607 2608 1532307317.518497 execve("/bin/date", ["date"], [...])
2608 2607 2608 1532307317.526037 write(1<pipe:[55019]>, "2018...", 43) = 43
2607 2603 2607 1532307317.515482 read(3<pipe:[55019]>, "2018...", 128) = 43
2607 2603 2607 1532307317.535603 write(1</home/.../b.txt>, "bb2018...", 45) = 45
\end{verbatim}
\end{scriptsize}
\end{frame}

\begin{frame}{REPTRACE Framework}
\begin{center}
\includegraphics[width=0.8\textwidth]{example-image}
\end{center}
\end{frame}

\begin{frame}{Dependency Graph Generation and Augmentation}
\begin{itemize}
\item Difference-induced dependency (DID)
\item Runtime-value-induced dependency (RID)
\item Filter noisy dependencies
\item Establish dependencies based on text similarity
\end{itemize}
\end{frame}

\begin{frame}{Graph-Traversal-based Causality Analysis}
\begin{itemize}
\item Root-cause localization starting from inconsistent artifacts
\item File-level localization for scripts and build commands
\item Use CLOEXEC flags to classify files
\end{itemize}
\end{frame}

\begin{frame}{Evaluation Results}
\begin{itemize}
\item RQ1: Command-level localization effective (A@1 0.6611, A@10 0.9000)
\item RQ2: DID reduces search scope, RID adds acceptable edges
\item RQ3: Not sensitive to parameter, best around [0.30; 0.70]
\item RQ4: Higher accuracy than RepLoc (Top-1 Accuracy 0.667)
\end{itemize}
\end{frame}

\begin{frame}{RepFix: Automated Patching}
\begin{itemize}
\item First study to generate patches for unreproducible builds automatically
\item Tracing-based fine-grained localization
\item History-based patch generation
\item Patch validation criteria
\end{itemize}
\end{frame}

\begin{frame}{RepFix Framework}
\begin{center}
\includegraphics[width=0.8\textwidth]{example-image}
\end{center}
\end{frame}

\begin{frame}{RepFix Evaluation}
\begin{itemize}
\item RQ1: Fixed 64 out of 116 packages
\item RQ2: Fine-grained localization and token-based patch generation effective
\item RQ3: Validation most time-consuming (avg 364.40s)
\item RQ4: Effectively solves real-world unreproducible build issues
\end{itemize}
\end{frame}

\begin{frame}{Conclusions}
\begin{itemize}
\item REPTRACE: Effective framework for root cause identification
\item RepFix: Automated patching for unreproducible builds
\item Both show promising results on real-world packages
\item Future work: More accurate dependency identification, automatic patching
\end{itemize}
\end{frame}

\begin{frame}
\centering
\Huge{Thank you!}
\par
\vspace{1cm}
\large{Q\&A}
\end{frame}

\end{document}

