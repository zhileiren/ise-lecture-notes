\documentclass{beamer}


\mode<presentation>
{
  \usetheme{CambridgeUS}
	\usecolortheme{beaver}
  % or ...

  \setbeamercovered{transparent}
  % or whatever (possibly just delete it)
}


\usepackage{xeCJK}
\usepackage{ulem}
\usepackage[english]{babel}
\usepackage[utf8]{inputenc}
\usepackage{times}
\usepackage[T1]{fontenc}
\usepackage{hyperref}
\usepackage{pifont}
\usepackage{biblatex}
\usepackage{bibentry}
\usepackage{verbatim}
\bibliography{cite}
\newcommand{\cmark}{\ding{51}}%
\newcommand{\xmark}{\ding{55}}%
\setCJKmainfont{WenQuanYi Micro Hei}
\renewcommand{\raggedright}{\leftskip=0pt \rightskip=0pt plus 0cm}
\raggedright

\let\oldfootnotesize\footnotesize
\renewcommand*{\footnotesize}{\oldfootnotesize\tiny}

\title[Intelligent Software Engineering] 
{Intelligent Software Engineering}
\subtitle{Introduction to Artificial Intelligence}

\author[Zhilei Ren] 
{Zhilei Ren}

\institute[Dalian University of Technology] % (optional, but mostly needed)
{
\\\includegraphics[width=0.1\textwidth]{../utils/logo.png}\\
Dalian University of Technology
}


\subject{Software Engineering}



\pgfdeclareimage[width=0.08\textwidth]{university-logo}{../utils/logo.png}
\logo{\pgfuseimage{university-logo}}



% Delete this, if you do not want the table of contents to pop up at
% the beginning of each subsection:
\AtBeginSubsection[]
{
  \begin{frame}<beamer>{Outline}
    \tableofcontents[currentsection,currentsubsection]
  \end{frame}
}


% If you wish to uncover everything in a step-wise fashion, uncomment
% the following command: 

%\beamerdefaultoverlayspecification{<+->}

\setbeamertemplate{section in toc}[circle]
\setbeamertemplate{items}[circle]
\setbeamertemplate{caption}[numbered]
\setbeamertemplate{bibliography item}{\insertbiblabel}
\setbeamertemplate{bibliography entry title}{}
\setbeamertemplate{bibliography entry journal}{}

\begin{document}

\begin{frame}
  \titlepage
\end{frame}

%\begin{frame}{Outline}
%  \tableofcontents[currentsection,currentsubsection, 
%    hideothersubsections, 
%    sectionstyle=show,
%]
%\end{frame}

\AtBeginSection[]
{
 \begin{frame}<beamer>
 \frametitle{Outline}
 \tableofcontents[currentsection]
 \end{frame}
}
\begin{frame}[t]{bug or feature?}
\end{frame}

\begin{frame}[t]{Search-Based Software Engineering}
    Search-based software engineering (SBSE) applies metaheuristic search techniques such as genetic algorithms, simulated annealing and tabu search to software engineering problems. Many activities in software engineering can be stated as optimization problems. Optimization techniques of operations research such as linear programming or dynamic programming are often impractical for large scale software engineering problems because of their computational complexity or their assumptions on the problem structure. Researchers and practitioners use metaheuristic search techniques, which impose little assumptions on the problem structure, to find near-optimal or ``good-enough'' solutions\footnote{\url{https://en.wikipedia.org/wiki/Search-based_software_engineering}}.
\end{frame}

\begin{frame}[t]{Mining Software Repositories}
Within software engineering, the mining software repositories (MSR) field analyzes the rich data available in software repositories, such as version control repositories, mailing list archives, bug tracking systems, issue tracking systems, etc. to uncover interesting and actionable information about software systems, projects and software engineering\footnote{\url{https://en.wikipedia.org/wiki/Mining_software_repositories}}.
\end{frame}

\begin{frame}[t]{Empirical Software Engineering}
Empirical software engineering (ESE) is a subfield of software engineering (SE) research that uses empirical research methods to study and evaluate an SE phenomenon of interest. The phenomenon may refer to software development tools/technology, practices, processes, policies, or other human and organizational aspects\footnote{\url{https://en.wikipedia.org/wiki/Empirical_software_engineering}}. Common research methods used in ESE for primary and secondary research are the following:

\begin{enumerate}
    \item Primary research (experimentation, case study research, survey research, simulations in particular software Process simulation)
    \item Secondary research methods (Systematic reviews, Systematic mapping studies, rapid reviews, tertiary review)
\end{enumerate}
\end{frame}

\end{document}

